
%\chapter{Introdução}

	\begin{flushright}
		\textit{``Não existe resposta errada: \\ ela apenas não responde a essa pergunta''.\\
		Confia}
	\end{flushright}
	
	Quando empurramos ou puxamos um determinado objeto tentando movê-lo, percebemos que existe certa dificuldade para coloca-lo em movimento. Essa dificuldade deve-se à força de atrito, que é uma força que se opõe ao movimento de objetos que estão sob a ação de uma força. Ela age paralelamente à superfície de contato em sentido contrario a força aplicada sobre um corpo\cite{da2019estudo}.
	
	Neste documento também são trazidas informações sobre o desenvolvimento e uso de roldanas e polias. Analogamente, desde os séculos passados, com o avanço da tecnologia, houve a necessidade de que o esforço físico fosse empregado cada vez menos ou com menor grau de peso resultante para que se pudesse ter um rendimento maior nas ações de transportes e elevações de materiais ou de objetos pesados. Com isso, as roldanas, também chamadas de polias, são tipos de rodas utilizados em um sistema para direcionar a força feita sobre determinados objetos por meio de fios, cordas ou cabos, de modo que seja possível desviar a trajetória ou até mesmo levantá-los. As roldanas podem facilitar a realização de algumas tarefas, dependendo da maneira com que elas são interligadas. Dessa forma, existem dois tipos de polias: as polias fixas e as polias móveis\cite{site1}.