%\chapter{Conclusão} - manter comentado
\label{Conclusão}


Após a análise  da pratica e dos tempos de execução em todas as arvores, é notável que, se tomado os mesmos métodos para inserir e buscar, a arvore balanceada é a que tem melhor desempenho entre as 3, seguindo por randômica e sequencial (A arvore sequencial só foi capaz de ser medida para inserção quando teve sua implementação da inserção modificada, e assim o seu tempo de execução foi reduzido drasticamente). Esses desempenhos se devem a organização das arvores, que demonstram que, quanto mais organizadas e balanceadas, melhor será o desempenho computacional de seus métodos. 



