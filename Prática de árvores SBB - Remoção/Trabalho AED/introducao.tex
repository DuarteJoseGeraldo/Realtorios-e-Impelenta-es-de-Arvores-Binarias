%\chapter{Introdução}
\label{Introdução}

%\textbf{Organização de informação não se dá da mesma forma em qualquer domínio.}
Esse trabalho foi desenvolvido em conjunto, feito por Raphael Gomes, José Geraldo e Pedro Arthur, no intuito de executar exclusão em Arvores SBB e Patricia\cite{nivioziviani} e na arvore Trie\cite{Samuellucas97}, porém editadas, adicionando métodos que permitissem a execução do trabalho. Essas implementações foram utilizadas para medir tempos de execução sobre a exclusão e busca de elementos de cada arvore.

Com o intuito de executar a atividade em diferentes ambientes, foram utilizadas 3 tipos de arvores diferentes para exclusão de 5\% dos elementos, e elas são: Arvore SBB com elementos sequenciais, Arvore SBB com elementos randômicos, e Arvore SBB balanceada com elementos sequenciais. Em teoria, deveriam ser utilizados arvores com os seguintes tamanhos:
        \begin{center}
        1. 10\textsuperscript{3} elementos\\
        2. 10\textsuperscript{5} elementos\\ 
        3. 10\textsuperscript{7} elementos\\
        4. 10\textsuperscript{9} elementos
        \end{center}

Já para a Trie e Patricia, as arvores geradas devem ser utilizadas para busca de 1\% dos elementos e medidos os tempos dessas buscas. Os tamanhos para a arvore são:
        \begin{center}
        1. 10\textsuperscript{3} elementos\\
        2. 10\textsuperscript{5} elementos\\ 
        3. 10\textsuperscript{7} elementos\\
        \end{center}
