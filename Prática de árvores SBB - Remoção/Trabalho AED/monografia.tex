%% abtex2-modelo-trabalho-academico.tex, v-1.8 laurocesar
%% Copyright 2012-2013 by abnTeX2 group at http://abntex2.googlecode.com/ 
%%
%% This work may be distributed and/or modified under the
%% conditions of the LaTeX Project Public License, either version 1.3
%% of this license or (at your option) any later version.
%% The latest version of this license is in
%%   http://www.latex-project.org/lppl.txt
%% and version 1.3 or later is part of all distributions of LaTeX
%% version 2005/12/01 or later.
%%
%% This work has the LPPL maintenance status `maintained'.
%% 
%% The Current Maintainer of this work is the abnTeX2 team, led
%% by Lauro César Araujo. Further information are available on 
%% http://abntex2.googlecode.com/
%%
%% This work consists of the files abntex2-modelo-trabalho-academico.tex,
%% abntex2-modelo-include-comandos and abntex2-modelo-references.bib
%%

% ------------------------------------------------------------------------
% ------------------------------------------------------------------------
% abnTeX2: Modelo de Trabalho Academico (tese de doutorado, dissertacao de
% mestrado e trabalhos monograficos em geral) em conformidade com 
% ABNT NBR 14724:2011: Informacao e documentacao - Trabalhos academicos -
% Apresentacao
% ------------------------------------------------------------------------
% ------------------------------------------------------------------------

\documentclass[
	% -- opções da classe memoir --
	11pt,				% tamanho da fonte
	openright,			% capítulos começam em pág ímpar (insere página vazia caso preciso)
	oneside,			% twoside para impressão em verso e anverso. Oposto a oneside
	a4paper,			% tamanho do papel. 
	% -- opções da classe abntex2 --
	%chapter=TITLE,		% títulos de capítulos convertidos em letras maiúsculas
	%section=TITLE,		% títulos de seções convertidos em letras maiúsculas
	%subsection=TITLE,	% títulos de subseções convertidos em letras maiúsculas
	%subsubsection=TITLE,% títulos de subsubseções convertidos em letras maiúsculas
	% -- opções do pacote babel --
	english,			% idioma adicional para hifenização
	french,				% idioma adicional para hifenização
	spanish,			% idioma adicional para hifenização
	brazil,				% o último idioma é o principal do documento
	]{abntex2}

\usepackage{abntex2-cefetmg-timoteo}

% ---
% PACOTES
% ---

% ---
% Pacotes fundamentais 
% ---
\usepackage{cmap}				% Mapear caracteres especiais no PDF
\usepackage{lmodern}			% Usa a fonte Latin Modern			
\usepackage[T1]{fontenc}		% Selecao de codigos de fonte.
\usepackage[utf8]{inputenc}		% Codificacao do documento (conversão automática dos acentos)
\usepackage{lastpage}			% Usado pela Ficha catalográfica
\usepackage{indentfirst}		% Indenta o primeiro parágrafo de cada seção.
\usepackage{color}				% Controle das cores
\usepackage{graphicx}			% Inclusão de gráficos
\PassOptionsToPackage{normalem}{ulem} % Para não usar sublinhado em referências bibliográficas
\usepackage{ulem}
\usepackage{multicol}
% ---

%Trocar fonte para Arial ou Helvetica
%\usepackage{uarial}
\usepackage{helvet}
\renewcommand{\familydefault}{\sfdefault}

		
% ---
% Pacotes adicionais, usados apenas no âmbito do Modelo Canônico do abnteX2
% ---
%\usepackage{lipsum}				% para geração de dummy text
% ---

% ---
% Pacotes de citações
% ---
\usepackage[brazilian,hyperpageref]{backref}	 % Paginas com as citações na bibl
\usepackage[alf,abnt-thesis-year=both,]{abntex2cite}	% Citações padrão ABNT


%Outros pacotes

\usepackage{longtable}


%\setlrmarginsandblock{3cm}{2cm}{*}
%\setulmarginsandblock{3cm}{2cm}{*}
%\checkandfixthelayout


% --- 
% CONFIGURAÇÕES DE PACOTES
% --- 

% ---
% Configurações do pacote backref
% Usado sem a opção hyperpageref de backref
\renewcommand{\backrefpagesname}{Citado na(s) página(s):~}
% Texto padrão antes do número das páginas
\renewcommand{\backref}{}
% Define os textos da citação
\renewcommand*{\backrefalt}[4]{
	\ifcase #1 %
		Nenhuma citação no texto.%
	\or
		Citado na página #2.%
	\else
		%Citado #1 vezes nas páginas #2.%
		Citado nas páginas #2.%
	\fi}%
% ---


% ---
% Informações de dados para CAPA e FOLHA DE ROSTO
% ---
\titulo{Arvores SBB, Trie e Patricia: Exclusão e Busca}
\autor{Raphael Gomes Wagner e José Geraldo Duarte Junior e Pedro Arthur Diniz}
\local{Timóteo}
\data{2022}
\orientador{Gustavo Martins}
\instituicao{%
  Centro Federal de Educação Tecnológica de Minas Gerais
  \par
  Campus Timóteo
  \par
  Graduação em Engenharia de Computação
}
\tipotrabalho{Lista 6 de AED}
% O preambulo deve conter o tipo do trabalho, o objetivo, 
% o nome da instituição e a área de concentração 

%\preambulo{Proposta de pesquisa apresentada à Coordenação de Engenharia de Computação do Campus Timóteo do Centro Federal de Educação Tecnológica de Minas Gerais para obtenção do grau de Bacharel em Engenharia de Computação.}

\preambulo{Trabalho 6 de Algoritmos e Estrutura de Dados sobre exclusão em Arvores Binarias no curso de Engenharia de Computação no Centro Federal de Educação Tecnológica de Minas Gerais}

% ---

% ---
% Configurações de aparência do PDF final

% informações do PDF
\makeatletter
\hypersetup{
     	%pagebackref=true,
		pdftitle={\@title}, 
		pdfauthor={\@author},
    	pdfsubject={\imprimirpreambulo},
	    pdfcreator={LaTeX with abnTeX2},
		pdfkeywords={abnt}{latex}{abntex}{abntex2}{trabalho acadêmico}, 
		colorlinks=true,       		% false: boxed links; true: colored links
    	linkcolor=black,          	% color of internal links
    	citecolor=black,        		% color of links to bibliography
    	filecolor=black,      		% color of file links
		urlcolor=black,
		bookmarksdepth=4
}
\makeatother
% --- 

% --- 
% Espaçamentos entre linhas e parágrafos 
% --- 

% O tamanho do parágrafo é dado por:
\setlength{\parindent}{1.3cm}

% Controle do espaçamento entre um parágrafo e outro:
\setlength{\parskip}{0.2cm}  % tente também \onelineskip

% ---
% compila o indice
% ---
\makeindex
% ---

% ----
% Início do documento
% ----
\begin{document}

% Retira espaço extra obsoleto entre as frases.
\frenchspacing 

% ----------------------------------------------------------
% ELEMENTOS PRÉ-TEXTUAIS
% ----------------------------------------------------------
 \pretextual

% ---
% Capa
% ---
\imprimircapa
% ---

% ---
% Folha de rosto
% (o * indica que haverá a ficha bibliográfica)
% ---
\imprimirfolhaderosto*
% ---

% ---
% Inserir a ficha bibliografica
% ---

% Isto é um exemplo de Ficha Catalográfica, ou ``Dados internacionais de
% catalogação-na-publicação''. Você pode utilizar este modelo como referência. 
% Porém, provavelmente a biblioteca da sua universidade lhe fornecerá um PDF
% com a ficha catalográfica definitiva após a defesa do trabalho. Quando estiver
% com o documento, salve-o como PDF no diretório do seu projeto e substitua todo
% o conteúdo de implementação deste arquivo pelo comando abaixo:
%
% \begin{fichacatalografica}
%		\includegraphics{inclusao-fichaCatalografica.pdf}
% \end{fichacatalografica}
%\begin{fichacatalografica}
%	\vspace*{\fill}					% Posição vertical
%	\hrule							% Linha horizontal
%	\begin{center}					% Minipage Centralizado
%	\begin{minipage}[c]{12.5cm}		% Largura
%	
%	\imprimirautor
%	
%	\hspace{0.5cm} \imprimirtitulo  / \imprimirautor. --
%	\imprimirlocal, \imprimirdata-
%	
%	\hspace{0.5cm} \pageref{LastPage} p. : il. (algumas color.) ; 30 cm.\\
%	
%	\hspace{0.5cm} \imprimirorientadorRotulo~\imprimirorientador\\
%	
%	\hspace{0.5cm}
%	\parbox[t]{\textwidth}{\imprimirtipotrabalho~--~\imprimirinstituicao,
%	\imprimirdata.}\\
%	
%	\hspace{0.5cm}
%		1. Palavra-chave1.
%		2. Palavra-chave2.
%		I. Orientador.
%		II. Universidade xxx.
%		III. Faculdade de xxx.
%		IV. Título\\ 			
%	
%	\hspace{8.75cm} CDU 02:141:005.7\\
%	
%	\end{minipage}
%	\end{center}
%	\hrule
%\end{fichacatalografica}
% ---

% ---
% Inserir errata
% ---
%\begin{errata}
%Elemento opcional da \citeonline[4.2.1.2]{NBR14724:2011}. Exemplo:

%\vspace{\onelineskip}

%FERRIGNO, C. R. A. \textbf{Tratamento de neoplasias ósseas apendiculares com
%reimplantação de enxerto ósseo autólogo autoclavado associado ao plasma
%rico em plaquetas}: estudo crítico na cirurgia de preservação de membro em
%cães. 2011. 128 f. Tese (Livre-Docência) - Faculdade de Medicina Veterinária e
%Zootecnia, Universidade de São Paulo, São Paulo, 2011.

%\begin{table}[htb]
%\center
%\footnotesize
%\begin{tabular}{|p{1.4cm}|p{1cm}|p{3cm}|p{3cm}|}
%  \hline
%   \textbf{Folha} & \textbf{Linha}  & \textbf{Onde se lê}  & \textbf{Leia-se}  \\
%    \hline
%    1 & 10 & auto-conclavo & autoconclavo\\
%   \hline
%\end{tabular}
%\end{table}

%\end{errata}
% ---

% ---
% Inserir folha de aprovação
% ---

% Isto é um exemplo de Folha de aprovação, elemento obrigatório da NBR
% 14724/2011 (seção 4.2.1.3). Você pode utilizar este modelo até a aprovação
% do trabalho. Após isso, substitua todo o conteúdo deste arquivo por uma
% imagem da página assinada pela banca com o comando abaixo:
%
%\includegraphics[width=1\textwidth]{inclusao-folhaDeAprovacao.pdf}
%\includegraphics[width=1\textwidth]{inclusao-ata.pdf}



.

.

.



%
% ---

% ---
% Dedicatória
% ---
\begin{dedicatoria}
    \vspace*{\fill}
	\begin{flushright}
		Dedico a\\
		Leandro Santana, Arthur Gomes, José Geraldo Duarte Junior\\ 
		Felipe Reggiane, Daniel Vidal, Guilherme Heringer, Nathan Teixeira e Pedro Arthur.
	\end{flushright}
\end{dedicatoria}
% ---

% ---
% Agradecimentos
% ---
\begin{agradecimentos}

Agradeço a todos da dedicação por terem passado em AED I comigo, e aos dois primeiros por terem ajudado no estudo da matéria, me ensinando.


\end{agradecimentos}

% ----------------------------------------------------------
% ELEMENTOS TEXTUAIS
% ----------------------------------------------------------
\textual

% ----------------------------------------------------------
% Introdução
% ----------------------------------------------------------
\chapter{Introdução}
\label{Introdução}

%\chapter{Introdução}

	\begin{flushright}
		\textit{``Não existe resposta errada: \\ ela apenas não responde a essa pergunta''.\\
		Confia}
	\end{flushright}
	
	Quando empurramos ou puxamos um determinado objeto tentando movê-lo, percebemos que existe certa dificuldade para coloca-lo em movimento. Essa dificuldade deve-se à força de atrito, que é uma força que se opõe ao movimento de objetos que estão sob a ação de uma força. Ela age paralelamente à superfície de contato em sentido contrario a força aplicada sobre um corpo\cite{da2019estudo}.
	
	Neste documento também são trazidas informações sobre o desenvolvimento e uso de roldanas e polias. Analogamente, desde os séculos passados, com o avanço da tecnologia, houve a necessidade de que o esforço físico fosse empregado cada vez menos ou com menor grau de peso resultante para que se pudesse ter um rendimento maior nas ações de transportes e elevações de materiais ou de objetos pesados. Com isso, as roldanas, também chamadas de polias, são tipos de rodas utilizados em um sistema para direcionar a força feita sobre determinados objetos por meio de fios, cordas ou cabos, de modo que seja possível desviar a trajetória ou até mesmo levantá-los. As roldanas podem facilitar a realização de algumas tarefas, dependendo da maneira com que elas são interligadas. Dessa forma, existem dois tipos de polias: as polias fixas e as polias móveis\cite{site1}.

\chapter{Desenvolvimento}
\label{Desenvolvimento}
%\chapter{Desenvolvimento}
\label{desenvolvimento}
\section{Inserção}
Como dito anteriormente, a implementação desta arvore segue a logica de que seriam inseridos arquivos de 1kB e de 256B, considerando também que o tamanho de um setor na unidade de armazenamento contém 512 Bytes, e uma página contém 4 kB.
Isso resultou em 2 modelos de arvores para cada tipo, ou seja 2 arvores B instanciadas para receber sucessivamente arquivos de 1kB em uma e arquivos de 256B na outra. O mesmo caso ocorre nas arvores B*.

Logo as características resultantes destas arvores foram:\\
\begin{center}
Arvore B - 1: m = 4;\\
Arvore B - 2: m = 8;\\
Arvore B* - 1: m = 4;\\
Arvore B* - 2: m = 8.
\end{center}

Os arquivos inseridos são representados por objetos do tipo MeuItem desenvolvido também pelo professor e pesquisador da UFMG, Nivio Ziviani. este objeto armazena um inteiro que foi gerado de forma pseudoaleatória utilizando de um algoritmo para esse fim.  


\subsection{Arvore B - Arquivos 1kB}

\begin{figure}[ht]
    \centering
    \includegraphics[scale=0.6]{Trabalho AED/fig/Planilha sem título - Página2.pdf}
    \label{fig:my_label}
\end{figure}
\begin{center}
        \begin{tabular}{| l | r |}
            \hline
            Quantidade de arquivos & Tempo em Milissegundos\\
            \hline
            10\textsuperscript{3} & 13\\
            10\textsuperscript{5} & 228\\
            10\textsuperscript{7} & 24395\\
            \hline
        \end{tabular}
    \end{center}
\subsection{Arvore B - Arquivos 256B}

\begin{figure}[ht]
    \centering
    \includegraphics[scale=0.6]{Trabalho AED/fig/Planilha sem título - Página4 (1).pdf}
    \label{fig:my_label}
\end{figure}
\begin{center}
        \begin{tabular}{| l | r |}
            \hline
            Quantidade de arquivos & Tempo em Milissegundos\\
            \hline
            10\textsuperscript{3} & 1\\
            10\textsuperscript{5} & 38\\
            10\textsuperscript{7} &  18951\\
            \hline
        \end{tabular}
    \end{center}


\subsection{Arvore B* - Arquivos 1kB}

\subsection{Arvore B* - Arquivos 256B}


\section{Remoção}

Após o processo de inserção daqueles objetos que representam os arquivos a serem inseridos na arvore, são executados testes de tempo de remoção de 5\% dos mesmos. Gerando os seguintes resultados a seguir:

\subsection{Arvore B - Arquivos 1kB}

\begin{figure}[ht]
    \centering
    \includegraphics[scale=0.6]{Trabalho AED/fig/Planilha sem título - Página6.pdf}
    \label{fig:my_label}
\end{figure}
 \begin{center}
        \begin{tabular}{| l | r |}
            \hline
            Quantidade de arquivos & Tempo em Nanossegundos\\
            \hline
            10\textsuperscript{3} & 264200\\
            10\textsuperscript{5} &  6761500\\
            10\textsuperscript{7} &  3771433700 \\
            \hline
        \end{tabular}
    \end{center}
\subsection{Arvore B - Arquivos 256B}

\begin{figure}[ht]
    \centering
    \includegraphics[scale=0.6]{Trabalho AED/fig/Planilha sem título - Página8.pdf}
    \label{fig:my_label}
\end{figure}
 \begin{center}
        \begin{tabular}{| l | r |}
            \hline
            Quantidade de arquivos & Tempo em Nanossegundos\\
            \hline
            10\textsuperscript{3} & 23200 \\
            10\textsuperscript{5} & 2424000\\
            10\textsuperscript{7} &  1348315800 \\
            \hline
        \end{tabular}
    \end{center}
\subsection{Arvore B* - Arquivos 1kB}

\subsection{Arvore B* - Arquivos 256B}

% ---
\chapter{Conclusão}
\label{Conclusão}
%\chapter{Conclusão} - manter comentado

	\begin{flushright}
		\textit{``Palavras não bastam, não dá pra entender\\
                    E esse medo que cresce não para\\
                    É uma história que se complicou\\
                    Eu sei bem o porquê ''\\Tiê}
	    \end{flushright}
	    
	    Por fim, por meio deste trabalho foi possível além de relembrar e aprender conceitos da Física, foi possível conhecer e aplicar as ferramentas do sistema de texto LaTex.
	    Com isso é possível sim enxergar utilidade pratica neste sistema principalmente para o fator formatação, onde o texto por possuir um padrão de escrita pré definido no sistema, se formata automaticamente de acordo com o desenvolvimento do mesmo.
	    De tal forma o preconceito para com este sistema de texto deixa de existir após este trabalho.



% ----------------------------------------------------------
% ELEMENTOS PÓS-TEXTUAIS
% ----------------------------------------------------------
\postextual


% ----------------------------------------------------------
% Referências bibliográficas
% ----------------------------------------------------------
\bibliography{bibfile}

% ----------------------------------------------------------
% Glossário
% ----------------------------------------------------------
%
% Consulte o manual da classe abntex2 para orientações sobre o glossário.
%
%\glossary

% ----------------------------------------------------------
% Apêndices
% ----------------------------------------------------------

% ---
% Inicia os apêndices
% ---
%\begin{apendicesenv}

% Imprime uma página indicando o início dos apêndices
%\partapendices

% ----------------------------------------------------------
%\chapter{Quisque libero justo}
% ----------------------------------------------------------

%\lipsum[50]

% ----------------------------------------------------------
%\chapter{Nullam elementum urna vel imperdiet sodales elit ipsum pharetra ligula
%ac pretium ante justo a nulla curabitur tristique arcu eu metus}
% ----------------------------------------------------------
%\lipsum[55-57]

%\end{apendicesenv}
% ---


% ----------------------------------------------------------
% Anexos
% ----------------------------------------------------------

% ---
% Inicia os anexos
% ---
%\begin{anexosenv}

% Imprime uma página indicando o início dos anexos
%\partanexos


%\end{anexosenv}

%---------------------------------------------------------------------
% INDICE REMISSIVO
%---------------------------------------------------------------------

%\printindex

\end{document}
