%\chapter{Conclusão} - manter comentado
\label{Conclusão}
Ao fim deste trabalho, foi abortado como foi realizada a remoção de elementos nas arvore SBB com elementos sequenciais, arvore SBB com elementos randômicos, e arvore SBB balanceada com elementos sequenciais e seus diversos tamanhos. Foi observado o tempo de execução da remoção de 5\% de elementos aleatórios em cada arvore SBB cujo o tamanho varia de 10\textsuperscript{3} elementos a 10\textsuperscript{7} elementos, pois como relatado anteriormente o Java não permite a criação de uma arvore de tamanho 10\textsuperscript{9}.Foi notado que dependendo do tamanho da arvore SBB e seu tipo de remoção seu tempo de execução varia, como pode ser observado no relatório o tempo de execução na remoção na SBB balanceada foi o menor dos três quando o seu tamanho era o menor entre as três arvores, mas em contrapartida seu tempo de execução quando é no maior tamanho é o pior entres as arvores e quem se sobre sai nesse quesito é a remoção na SBB sequencial. Outro tópico abortado foi a  realização de busca de elementos nas arvores TRIEs e nas arvores Patricia cujo foi observado os seus tempos de execução e seus desvios padrões 



