
%\chapter{Introdução}

	\begin{flushright}
		\textit{``Não existe resposta errada: \\ ela apenas não responde a essa pergunta''.\\
		Confia}
	\end{flushright}
	
	Um gerador de número pseudo-aleatório é um algoritmo derivado de uma função matemática que gera uma seqüência de números, os quais são aparentemente independentes um dos outros. O resultado da maioria dos geradores de números aleatórios não é verdadeiramente aleatório, eles somente possuem algumas das propriedades dos números aleatórios.
	
	Alguns exemplos de números aleatórios são: tempo de resposta de requisições de leitura de um disco rígido e tubos de descarga de gás. Número pseudo-aleatórios são uma parte crítica da computação moderna, da criptografia até o método de Monte Carlo passando por sistemas de simulação. Uma cuidadosa análise matemática é necessária para assegurar que a geração dos números seja suficientemente "aleatória"\cite{wiki1}.
	
	No desenvolver deste trabalho serão apresentados processos e resultados de testes de um dos mais conhecidos geradores de números aleatórios, que é baseado no chamado método das congruências lineares, apresentando dados que comprovam a pseudo aleatoriedade deste método e aplicação do mesmo em um algoritmo de preenchimento de vetores em Java.